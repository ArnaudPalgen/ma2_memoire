\subsection*{Old physique}
    La couche physique définit le format des adresses et trames LoRaMAC, ainsi que que le driver 
    permettant de simplifier l'utilisation du RN2483 par l'abstraction des communications UART.

    La classe \texttt{LoraAddr} qui définit les adresses LoRaMAC et possède deux attributs entier 
    qui sont le préfixe de l'adresse et le \textit{node-id}. Cette classe possède également une 
    fonction \texttt{toHex()} qui permet de sérialiser une adresse. C'est à dire de la convertir en 
    une chaine de caractères hexadecimal qui est le format de donnée demandé par le RN2483.

    La classe \texttt{LoraFrame} qui définit une trame LoRaMAC possède comme attributs tous les 
    champs d'une trame LoRaMAC. C'est à dire, les adresses source et destination, la commande MAC 
    dont le type choisi est une énumération, la payload, le numéro de séquence et deux booléens 
    pour les flags \textit{k} et \textit{has\_next}. Comme la classe \texttt{LoraAddr}, elle 
    possède une fonction \texttt{toHex()} mais également une fonction \texttt{build(data: str)} qui 
    retourne une instance d'une \texttt{LoraFrame} construite sur base d'une chaine de caractères hexadécimal.

    Ces deux classes utilisent l'annotation \texttt{@dataclass} qui génère des fonctions spéciales 
    comme \texttt{\_\_init\_\_()} et \texttt{\_\_repr\_\_()} automatiquement à partir d'une suite 
    d'attributs en utilisant les annotations de type. Pour la classe \texttt{LoraAddr}, le 
    paramètre de l'annotation, \texttt{frozen} est mis à \textit{True} pour rendre cette classe 
    immuable.

    Le driver du RN2483 est implémenté par la classe \texttt{LoraPhy} qui ne possède que des 
    attributs destinés à un usage interne. Son constructeur nécéssite néanmoins deux paramètres: le 
    baudrate et le port de la connection série.

    La première fonction qui doit être appellée avant tout interraction avec le RN2483 est la 
    fonction \texttt{init()}. Cette fonction ouvre la connexion série en instanciant un la classe 
    \texttt{Serial} de la librairie python \textit{pyserial} (version 3.4), démarre deux Threads 
    utilisés pour les communications entrantes et sortantes et envoi les deux premières trames UART 
    permettant de désactiver le protocole MAC, LoRaWAN, présent sur le RN2483 et de définir la 
    fréquence radio utilisée.

    L'envoi d'une trame UART se fait par la fonction interne \texttt{\_send\_phy(data)}. Cette fonction ajoute la trame UART passée en paramètre dans un buffer d'envoi s'il n'est pas complètement rempli. 
    Une trame UART est définie par la classe \texttt{UartFrame}, également avec l'annotation \texttt{@dataclass}, et possède trois attributs:
    \begin{itemize}
        \item \texttt{expected\_response} est de liste de réponses attendues en retour de l'envoi de cette trame. Les réponses possibles du RN2483 sont reprises dans une énumération. 
        \item \texttt{cmd} est la command UART qui est également une énumération. Une commande UART est par exemple, \textit{"mac pause"} pour désactiver LoRaWAN ou encore \textit{"radio tx"} pour transmettre des données.
        \item \texttt{data} qui est une chaîne de caractères représentant les données suivant la commande UART. Cette chaîne de craractère peut être vide pour certaines commandes comme \textit{"mac pause"}.
        L'assemblage de \texttt{cmd} et \texttt{data} donne une trame UART. Par exemple pour transmettre les données \textit{"AB16FD"}, la command UART est \textit{radio tx AB16FD"}
    \end{itemize}
    \vspace{0.5cm}

    Une fois qu'une trame a été ajoutée au buffer d'envoi, le Thread \texttt{\_uart\_tx} va se charger de 
    son envoi. Le buffer d'envoi (comme le buffer de réception) est une \texttt{Queue} FIFO de la librairie python \texttt{queue}.
    
    Pour récupérer une trame, il faut que une réponse correcte de la trame précédente ai été reçue ou que la trame a envoyée soit la première.
    Pour déterminer si une trame peut être envoyée le Thread vérifie que la connexion série existe 
    et que la valeur l'attribut du driver, \texttt{\_can\_send} soit \textit{True}. Dans le cas 
    contraire, le Thread est bloqué par une variable conditionelle \texttt{\_can\_send\_cond} également attribu du driver.
    La trame à envoyée est récupérée via une fonction \texttt{get} dont le paramètre \texttt{block} est 
    mis à \textit{True}, ce qui bloque le Thread quand aucune donnée n'est disponible dans le buffer et évite donc de consommer des ressources. Une fois la trame envoyée, un attribut du driver, \texttt{\_can\_send} est mis à \textit{False}.

    Le Thread \texttt{\_uart\_rx} lis les données de la connexion série via la fonction
    \texttt{readline()} de la classe \texttt{Serial}. Cette méthode bloque le Thread jusqu'au 
    moment où une ligne peut être lue sur la connexion série. Une fois la ligne reçue, la fonction 
    \texttt{\_process\_response} retourne \textit{True} si la réponse fait partie de celles 
    attendues et \textit{False} si ce n'est pas le cas. Cette fonction s'occupe égalemet d'ajouter 
    au buffer de réception une trame LoRaMAC si la réponse UART commence par \textit{"radio\_rx"}. 
    Si \texttt{\_process\_response} retourne 
    \textit{True}, le Thread met la valeur de \texttt{\_can\_send} à \textit{True} et lui notifie que la valeur de cette variable a changée via la variable conditionelle \texttt{\_can\_send\_cond}. Le Thread \texttt{\_uart\_tx} peut ainsi envoyer la prochaine trame UART s'il y en a une.

    Les fonctions destinées à un usage externe sont les suivantes:
    \begin{itemize}
        \item \texttt{phy\_send(loraFrame: LoraFrame)}: Permet d'envoyer une trame LoRaMAC
        \item \texttt{phy\_timeout(timeout: int)}: Permet de définir le temps d'expiration du watchdog timer qui est activé pour chaque transmission ou réception. Pour le désactiver, la valeur qui doit être utilisée est zéro.
        \item \texttt{phy\_rx()}: Permet de mettre le RN2483 en mode réception
        \item \texttt{getFrame()}: Permet d'obtenir la prochaine trame du biffer de réception. Si le buffer est vide, la fonction est bloquante.
    \end{itemize}