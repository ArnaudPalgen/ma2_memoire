\section{Topologie}\label{sec:archi-topologie}
\renewcommand{\rightmark}{Topologie}

Cette section décrit et motive la topologie et les types de noeuds du réseau hybride IEEE 802.15.4-LoRa de capteurs. Cette topologie a été choisie pour correspondre au mieux au cas d'utilisation décrit dans l'introduction.\todo{REF INTRO}

\subsection*{Types de noeuds}
    Le protocole élaboré pour ce réseau hybride défini 3 types de noeuds:
    \begin{itemize}
        \item[-] La \textbf{racine LoRa} est la passerelle entre le réseau et un réseau IP externe. Elle possède donc une interface LoRa et par exemple une interface Wi-Fi.
        \item[-] La \textbf{racine RPL}, comme son l'indique, est la racine d'un réseau RPL. Elle possède également une interface LoRa pour communiquer avec la racine LoRa.
        \item[-] Les \textbf{noeuds RPL} sont des noeuds du réseau RPL.
    \end{itemize}

\subsection*{Topologie}
Comme l'illustre la figure~\ref{fig:archi-topologie} la topologie choisie est une topologie mixte.

On a d'abord un ensemble de réseaux RPL possèdant chacun un préfixe IP attribué par la racine LoRa (par exemple $0x02$ ou $0x04$ sur la figure~\ref{fig:archi-topologie}), dans lesquels les communications sont réalisées par des liens IEEE 802.15.4.
De part l'utilisation de RPL, chaque réseaux forme un DODAG.

Ensuite, chaque réseau RPL est connecté à la racine LoRa via sa racine. Les liens entre les racines RPL et la racine LoRa forment une topologie en étoile.

Ce choix de topologie offre plusieurs avantages: Le fait d'avoir un ensemble de réseaux RPL implique qu'il n'y a pas qu'une seule racine RPL et donc pas qu'un seul point de défaillance; L'administration du réseau est également simplifiée. En effet, un réseau RPL pourraît, par exemple, correspondre à une parcelle de terrain. 

L'inconvéninent de cette topologie est que la racine LoRa constitue un seul point de défaillance. Une solution plus robuste qui a été envisagée, est un réseau maillé pour les communications LoRa. Cependant, une telle architecture est difficile à mettre en oeuvre car le protocole MAC a établir pour les liens LoRa est plus complexe et nécessite un protocole de routage.

\begin{figure}[H]
    \centering
    \includegraphics[scale=0.7]{res/pictures/loramac-topologie.drawio.png}
    \caption{Topologie du réseau hybride.}
    \label{fig:archi-topologie}
\end{figure}

Ainsi, l'objectif de cette topologie est de transmettre les paquets IP à destination de la racine LoRa.