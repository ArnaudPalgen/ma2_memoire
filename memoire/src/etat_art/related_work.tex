\section{Projets similaires}\label{sec:etat_art-related-work}
\renewcommand{\rightmark}{Projet similaires}

% le truc principal du projet est de concevoir un protocol mac pour orchester les communications LoRa. Le protocole décrit en ...(archi) s'inspire du protocole AloHa. Cela peut entrainer des collisions. Pour minimiser ces collisions, Yi Chu, Paul D. Mitchell et David Grace proposent dans leur article lora/chu2012.pdf un ptorocole MAC nommé  ALOHA-QIR. Ce protocole combine Slotted ALOHA et le Q-Learning qui est une méthode d'apprentissage par renforcement. D'après les auteurs de l'article, les simulations réalisées indiquent que leur protocole atteint plus du double du débit de slotted Aloha et que le débit de bout en bout ainsi que l'efficacité énergétique sont significativement améliorés.

%Cependant, d'autres acrhitectures pour l'orechestration des communications sont possibles, comme l'utilisation d'un réseau MESH (todo décrire) rendue possible via l'adaptation pour LoRa du protocole MAC TSCH




% 
%lora/bezunartea2019.pdf done
%lora/chu2012.pdf
%lora/sartori2017.pdf
%lora/sensors-20-06893-v2 (1).pdf
%lora/WF-IoT.pdf
%LoRa_and_802.15.4/icii19.pdf
%%lora/bezunartea2019.pdf lora/chu2012.pdf lora/LoRa\ IoT\ Monitoring\ of\ Urban\ Tree.pdf lora/sartori2017.pdf lora/sensors-20-06893-v2\ \(1\).pdf lora/WF-IoT.pdf LoRa_and_802.15.4/icii19.pdf