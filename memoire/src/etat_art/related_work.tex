\section{Projets similaires}\label{sec:etat_art-related-work}
\renewcommand{\rightmark}{Projet similaires}

Une grande partie du travail de ce mémoire consiste en l'élaboration d'un protocole MAC pour
orchester les communications LoRa. Le protocole décrit au \autoref{chap:archi} s'inspire du
protocole AloHa qui peut entrainer des collisions. Pour les minimiser, Yi Chu, Paul D. Mitchell et
David Grace proposent dans leur article~\cite{6328420} un protocole MAC nommé ALOHA-QIR. Ce
protocole combine Slotted ALOHA et le Q-Learning qui est une méthode d'apprentissage par
renforcement. D'après les auteurs de l'article, les simulations réalisées indiquent que leur
protocole atteint plus du double du débit de slotted Aloha et que le débit de bout en bout ainsi
que l'efficacité énergétique sont significativement améliorés.


Cependant d'autres acrhitectures de réseaux LoRa sont possibles, comme un réseau multi-sauts
fonctionnant sur une adaptation de TSCH\ref{subsec:etat_art-802.15.4.tsch} proposée par Maite Bezunartea et ses co-auteurs dans leur article~\cite{8847137}. Leur solution, implémentée dans Contiki-NG\ref{subsec:etat_art:rtos:contiki}, permet d'utiliser le protocole de routage RPL\ref{sec:state-rpl} avec le \textit{spreading factor} de LoRa fixe (c.f.~\ref{sec:etat_art-lora}). Les mécanismes de sauts de canaux et de division du temps en slots permettent d'obtenir des transmissions de données réussies tout en restant efficace en terme d'énergie.

Pour obtenir un \textit{spreading factor} adapté à chaque lien LoRa, B.Satori et ses co-auteurs proposent dans leur article~\cite{8115756} un protocole MAC nommé RLMAC qui permet de sélectionner le \textit{spreading factor} optimal pour chaque lien et également une adaptation de la fonction objectif OF0~\ref{subsec:etat_art:rpl:Of0} permettant de sélectionner un chemin qui minimise le temps de transmission ce qui permet de diminuer la consommation en énergie.

%sartori2017 propose aussi d'utiliser RPL mais avec adaptation du SF

% 
%lora/bezunartea2019.pdf done
%lora/chu2012.pdf done
%lora/sartori2017.pdf done
%lora/sensors-20-06893-v2 (1).pdf
%lora/WF-IoT.pdf
%LoRa_and_802.15.4/icii19.pdf
%%lora/bezunartea2019.pdf lora/chu2012.pdf lora/LoRa\ IoT\ Monitoring\ of\ Urban\ Tree.pdf lora/sartori2017.pdf lora/sensors-20-06893-v2\ \(1\).pdf lora/WF-IoT.pdf LoRa_and_802.15.4/icii19.pdf