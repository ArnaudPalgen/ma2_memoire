\section{LoRa}\label{sec:etat_art-lora}
\renewcommand{\rightmark}{LoRa}

LoRa (Long Range)~\cite{sx1276:datasheet}\cite{paper:lora-reverse-engineering} est une technologie radio  fonctionnant sur une modulation propriétaire détenue par Semtech et basée sur \textit{chirp spread spectrum (CSS)}. Cette modulation consiste en un signal d'une amplitude constante pour lequel la fréquence diminue ou augmente au cours du temps. Ces variations qui sont pour LoRa linéaires, sont respectivement appelées \textit{downchirp} et \textit{upchirp}. Cette modulation permet d'obtenir des communications longues portées et basses énergie.

%\begin{wrapfigure}{r}{0.3\textwidth}
%    \includegraphics[scale=0.25]{res/pictures/lora-sf.png}
%    \caption{Signal en fonction du SF.\todo{ref}}
%    \label{fig:state-sf}
%\end{wrapfigure}
Une communication LoRa dépend des paramètres suivants:
\begin{itemize}
    \item \textbf{Spreading Factor (SF)} est une variable qui définit la vitesse de balayage du signal radio
    %i.e. le nombre chirps utilisés pour la transmission d'un symbole qui est $2^{SF}$
    (Fig~\ref{fig:state-sf}). Le spreading factor est compris entre 7 et 12. Plus sa valeur est grande, plus la vitesse de balayage est petite. Ceci implique un signal plus facile à décoder, mais avec un débit de données plus faible. Donc, inversement à une petite valeur de SF, une grande valeur de SF a un débit plus faible et une QoS (Quality of Service) plus élevée.
    
    \item \textbf{Bandwidth (BW)} détermine la largeur de la bande passante. Une bande passante plus large augmente la qualité du signal. Pour LoRa, en Europe, les valeurs de BW sont limitées à 125 KHz et 250KHz~\cite{lora-frequencyplan}.
    
    \item \textbf{Coding rate (CR)} détermine la quantité de bits redondants ajoutés par le codage d'erreur cylique utilisé par LoRa pour la détection et la correction d'erreurs. Cette méthode de correction d'erreur permet au signal de supporter de courtes interférences. Ainsi il possible de coder des données de 4 bits avec des redondances de 5, 6, 7 ou 8 bits. La valeur du CR se note donc 4/5, 4/6, 4/5 ou 4/8.
    
\end{itemize}

\begin{figure}[H]
    \centering
    \includegraphics[scale=0.25]{res/pictures/lora-sf.png}
    \caption{Spectogramme des différents Spreading Factors~\cite{ghoslya}.}
    \label{fig:state-sf}
\end{figure}


%La relation entre SF et BW est décrite par l'équation~\ref{eq:state-lora-rs}.
%\begin{equation}\label{eq:state-lora-rs}
%    Rs = \frac{BW}{2^{SF}}
%\end{equation}

\vspace{1cm}
Le format des paquets LoRa (Fig.~\ref{fig:state-lora-frame-format}) peut être explicite ou implicite.
Dans le mode implicite, le header n'est pas inclus dans le paquet. Pour cela, la taille de la payload, le coding rate et la présence de la Payload CRC doivent être configurés.
Ce mode permet de réduire la taille du paquet et donc du temps de transmission.

\begin{figure}[H]
    \centering
    \includegraphics[scale=0.6]{res/pictures/lora-frame-format.drawio.png}
    \caption{Format d'un paquet LoRa.}
    \label{fig:state-lora-frame-format}
\end{figure}

La datasheet du SX1276 intégré dans le RN2383, utilisé comme transceiver LoRa pour ce projet (c.f.~\ref{hardware:rn2483}), définit le temps d'émission d'un paquet LoRa reprit par l'équation~\ref{eq:state-lora-tframe}.
\begin{equation}\label{eq:state-lora-tframe}
    T_{frame} = T_{preamble} + T_{payload}
\end{equation}

Où $T_{preamble}$ et $T_{payload}$ sont respectivement le temps d'émission du preamble et du payload définis aux équations \ref{eq:state-lora-tpreamble} et \ref{eq:state-lora-tpayload}.

\begin{equation}\label{eq:state-lora-tpreamble}
    T_{preamble} = (n_{preamble} + 4,25)T_{sym}
\end{equation}

\begin{equation}\label{eq:state-lora-tpayload}
    T_{payload} = n_{payload} * T_{sym}
\end{equation}

Où $T_{sym} = \frac{1}{R_s}$, $n_{preamble}$ et $n_{payload}$ sont le nombre de symboles du preamble et du payload. $n_{preamble}$ est une valeur paramétrable du module radio et $n_{payload}$ est défini par l'équation~\ref{eq:state-lora-npayload}.

\begin{equation}\label{eq:state-lora-npayload}
    \begin{split}
    n_{payload} =8 +max
     \left( ceil \left[ \frac{8PL - 4SF + 28 + 16CRC - 20IH}{4(SF-2DE)} \right] (CR+4), 0 \right)
    \end{split}
\end{equation}

où:
\begin{itemize}
    \item $PL$ est le nombre d'octets de la payload (1 à 255)
    \item $IH=0$ si le header est présent, $IH=1$ sinon
    \item $DE=1$ si $LowDataRateOptimize=1$, $DE=0$ sinon.\\
    $LowDataRateOptimize$ permet d'optimiser la robustesse d'une transmission quand celle-ci est longue.
    \item $CR$ prend la valeur 1 pour $4/5$, 2 pour $4/6$, etc
\end{itemize}
